\documentclass[final]{beamer}

% ==================== Packages ====================
\usepackage[T1]{fontenc}
\usepackage[utf8]{inputenc}
\usepackage{lmodern}

\usepackage[size=custom, width=120,height=90, scale=1.1]{beamerposter}
\usetheme{gemini}
\usecolortheme{uoft}
\usefonttheme{serif}

\usepackage{graphicx}
\usepackage{grffile} % REQUIRED for spaces in filenames
\usepackage[export]{adjustbox}

\usepackage{tikz}
\usetikzlibrary{shapes,arrows,calc,positioning}

\usepackage{array}
\usepackage[usenames,dvipsnames]{xcolor}
\usepackage{booktabs}
\usepackage{ragged2e}
\usepackage{enumitem}

% ==================== Typography ====================
\usefonttheme{professionalfonts}

\setbeamerfont{headline title}{family=\rmfamily, size=\fontsize{80}{90}\selectfont}
\setbeamerfont{headline author}{family=\rmfamily, size=\fontsize{42}{48}\selectfont}
\setbeamerfont{headline institute}{family=\rmfamily, size=\fontsize{36}{42}\selectfont}

\setbeamerfont{block title}{family=\rmfamily, size=\fontsize{48}{60}\selectfont}
\setbeamerfont{block body}{size=\fontsize{30}{36}\selectfont}
\setbeamerfont{footline}{family=\rmfamily, size=\fontsize{28}{30}\selectfont}


% ==================== Colors ====================
\definecolor{uoftblue}{RGB}{139,183,223}
\setbeamercolor{headline}{bg=uoftblue, fg=white}
\setbeamercolor{footline}{bg=uoftblue, fg=white}
\setbeamercolor{block title}{bg=white, fg=uoftblue}
\setbeamercolor{structure}{fg=uoftblue}

% ==================== Column widths ====================
\newlength{\sepwidth}
\newlength{\colwidth}

\setlength{\sepwidth}{0.02\paperwidth}
\setlength{\colwidth}{0.30\paperwidth}

\newcommand{\separatorcolumn}{\begin{column}{\sepwidth}\end{column}}

% ==================== Tables ====================
\setlength{\tabcolsep}{10pt}
\renewcommand{\arraystretch}{1.3}

% ==================== Title ====================
\title{5-Year Outcomes of Total Ankle Replacement
with the Cadence Total Ankle Replacement System}

\author{
Zuhal Olomi\textsuperscript{1,2},
Mansur Halai\textsuperscript{1,2},
Ellie Pinsker\textsuperscript{1,2},
Ryan Khan\textsuperscript{2},
Timothy R. Daniels\textsuperscript{1,2}
}

\institute{
\textsuperscript{1}University of Toronto, Department of Surgery \quad \\
\textsuperscript{2}Unity Health Toronto, St. Michael's Hospital
}

% ================== Logo ================

\logoright{%
  \includegraphics[
    width=20cm,
    keepaspectratio
  ]{radiographic images/logo.png}%
}

% ==================== Footer ====================
\footercontent{
Department of Orthopaedic Surgery \hfill
St. Michael's Hospital, Unity Health Toronto \hfill
Undergraduate Research Symposium, 2026
}

% ==================== Document ====================
\begin{document}
\begin{frame}[t]

\begin{columns}[t]
\separatorcolumn

% ================= LEFT COLUMN =================
\begin{column}{\colwidth}

\begin{block}{Background}
\justifying
\Large
The ankle-joint is mostly subjected to more weight-bearing force per centimeter and enduring injuries than any other joint in the body. There exist various mechanical, biochemical, and anatomical support systems that account for the resilience to the processes of aging and trauma\textsuperscript{1}. Although ankle arthrodesis has always been the standard surgical intervention, ankle arthroplasty, or total ankle replacement (TAR), has gained notorious acceptance as the motion-preserving alternative. Improvements in the implant design and surgical technique have contributed to better survivorship and functional outcomes in modern TAR systems. 

\end{block}

\centering
\begin{tabular}{cc}
\includegraphics[height=0.32\linewidth]{cadence.jpeg}
\hspace{1cm}
\includegraphics[height=0.32\linewidth]{cadencemodel.png}
\end{tabular}

\vspace{0.4cm}
{\normalsize \textbf{Figure 1.} A 3-dimensional model of the prosthetic Cadence implant both isolated and inside an ankle joint.}

\begin{block}{Objective}
\justifying
\Large
The study aims to evaluate the 5-year clinical, and radiographic, and clinical outcomes of TAR using the Cadence\textsuperscript{TM} Total Ankle Replacement System, while accounting for valgus and varus deformities. 

\end{block}

\begin{block}{Study Flow (Consort Diagram)}
\centering

\end{block}
\includegraphics[height=19cm]{figures/consort.png}


\end{column}

\separatorcolumn

% ================= MIDDLE COLUMN =================
\begin{column}{\colwidth}

\begin{block}{Methodology}
\justifying
\Large
The Cadence\textsuperscript{TM} TAR System is a two-component, fixed-bearing, semi-constrained implant approved for clinical use in 2016\textsuperscript{2}. Patients undergoing primary TAR for end-stage ankle arthritis were prospectively followed for a minimum of five years. Clinical outcomes, radiographic alignment parameters, complications, and revision events were systematically collected and analyzed. Patient clinical data and radiographic measurements were obtained preoperatively and at a minimum of five years postoperatively and were compared longitudinally.

\end{block}

\begin{block}{Radiographic Measurements}
\begin{itemize}[leftmargin=*, itemsep=3mm]
\large{
\item — Talar tilt
\item — Tibiotalar angle range
\item — Joint congruence
\item — Tibial--ankle surface angle
\item — Tibiotalar surface angle
\item — Hindfoot alignment angle (pre-operative)
\item — Hindfoot alignment ratio (pre-operative)
\item — Hindfoot alignment distance (mm)
}
\end{itemize}
\end{block}

\centering
\begin{tabular}{cc}
\includegraphics[height=0.32\linewidth]{left lateral.jpg}
\hspace{1cm}
\includegraphics[height=0.32\linewidth]{oneankle.jpg}
\end{tabular}

\includegraphics[width = 34cm]{deformities.png}

\vspace{0.4cm}
{\normalsize \textbf{Figure 2.} a) Normal ankle, b) Ankle varus deformity, and c) Ankle valgus deformity.}


\end{column}

\separatorcolumn

% ================= RIGHT COLUMN =================
\begin{column}{\colwidth}

\begin{block}{Radiographic Assessments}
\centering
\begin{tabular}{ccc}

% ---------- TOP ROW (2 IMAGES) ----------
\multicolumn{3}{c}{
\includegraphics[height=0.32\linewidth]{radiographic images/normal_healthy_ankle.jpg}
\hspace{1cm}
\includegraphics[height=0.32\linewidth]{radiographic images/pre_tar.jpg}
} \\[6mm]

% ---------- BOTTOM ROW (3 IMAGES) ----------
\includegraphics[height=0.29\linewidth]{radiographic images/pre_tar_right.jpg} &
\includegraphics[height=0.29\linewidth]{radiographic images/lateral_view.jpg} &
\includegraphics[height=0.29\linewidth]{radiographic images/post_tar.jpg}

\end{tabular}



\vspace{0.4cm}
{\normalsize \textbf{Figure 3.} Radiographic assessments taken from patients pre-operatively and post-operatively. Image on top right demonstrates a valgus deformity on right ankle.}
\end{block}

\begin{block}{Conclusion}
\justifying
\Large
This study aims to provide mid-term clinical and radiographic outcome data for the Cadence™ TAR System. These findings will contribute to the growing body of evidence supporting total ankle arthroplasty as a motion-preserving alternative for patients with end-stage ankle arthritis.
\end{block}

\begin{block}{Clinical Significance}
\justifying
\Large
As TAR continues to evolve, evaluation of newer implant systems is critical to inform surgical decision-making and patient counseling. Mid-term outcome data contribute to understanding implant performance, alignment correction, and durability in patients with end-stage ankle arthritis.
\end{block}

\begin{block}{References}
\justifying

\begin{itemize}[leftmargin=*, itemsep=2mm]
\item[1.] Thomas RH, Daniels TR. Ankle arthritis. \textit{J Bone Joint Surg Am}. 2003;85(5):923–936.
\item[2.] Kooner S, Kayum S, Pinsker E, et al. Two-year outcomes after total ankle replacement with a novel fixed-bearing implant. \textit{Foot Ankle Int}. 2021;42(8):1002–1010.
\item[3.] Morash J, Walton DM, Glazebrook M. Ankle arthrodesis versus total ankle arthroplasty. \textit{Foot Ankle Clin}. 2017;22(2):251–266.
\end{itemize}

\end{block}


\end{column}

\separatorcolumn
\end{columns}

\end{frame}
\end{document}
